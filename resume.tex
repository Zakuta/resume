\documentclass[11pt,a4paper]{moderncv}

\usepackage[utf8]{inputenc}
\usepackage{multicol}     %% Allows to create more than two columns
\usepackage{array}        %% For adjusting column widths
\usepackage{color}
\usepackage{enumitem}     %% For eliminating space before itemize
\usepackage{microtype}
\usepackage[margin=0.2in, top=0.5in, bottom=0.1in]{geometry}
\usepackage{xcolor}
\usepackage[unicode]{hyperref}

\hypersetup{colorlinks=true, linkcolor=blue, urlcolor=cyan}

\newcolumntype{L}[1]{>{\raggedright\let\newline\\\arraybackslash\hspace{0pt}}m{#1}}
\newcolumntype{C}[1]{>{\centering\let\newline\\\arraybackslash\hspace{0pt}}m{#1}}
\newcolumntype{R}[1]{>{\raggedleft\let\newline\\\arraybackslash\hspace{0pt}}m{#1}}

\setlength{\parindent}{0pt} %% no indents in paragraph

\setlist{nosep}
\pagenumbering{gobble}

\moderncvstyle{banking}
\moderncvcolor{blue}
\moderncvicons{awesome}

\firstname{Yash}
\familyname{Srivastav}
\address{Sophomore}{Computer Science and Engineering}{IIT Kanpur}
\phone[mobile]{+91~7054133662}
\email{yashsriv@iitk.ac.in}
\homepage{cse.iitk.ac.in/users/yashsriv}
\social[github]{yashsriv}
\photo[64pt][0.4pt]{picture}

\newcommand{\education}[5]{
  \textbf{\large{#1}} \hfill\textit{#2}\\
  #5 \hfill \large{#3 : \textbf{#4}}
}
\newcommand{\neducation}[5]{
  & #1 & #2 & &#5 & #3 : #4
}

\newcommand{\achievement}[2]{
  \item #1 \hfill \textit{#2}
}
\newcommand{\nachievement}[3]{
  & #1 & #2 & #3
}

\newcommand{\experience}[3]{
  \item \textbf{\large{#1}} \hfill \emph{#3} \hfill \textit{#2}
}
\renewcommand{\familydefault}{pag}
\linespread{1.1}
\begin{document}
  \makecvtitle
  \section{Educational Qualifications}
  \begin{tabular}{L{0.05cm} L{6cm} l L{0.05cm} C{8cm} r}
      \neducation{B.Tech, CSE}{July'15-Present}{CPI}{9.12}{IIT Kanpur}\\
      \neducation{AISSCE - CBSE}{2015}{}{96.6\%}{Birla High School, Kolkata}\\
      \neducation{ICSE - CISCE}{2013}{}{96.6\%}{AG Church School, Kolkata}\\
    \end{tabular}
  \section{Academic Achievements and Scholarships}
  \begin{tabular}{L{0.05cm} L{6cm} C{8cm} l}
    \nachievement{JEE Advanced}{2015}{AIR \textbf{105}}\\
    \nachievement{JEE Mains}{2015}{AIR \textbf{288}}\\
    \nachievement{NSEC}{2015}{Qualified}\\
    \nachievement{KVPY}{2015}{AIR \textbf{12}}
  \end{tabular}
  \section{Projects}
    \begin{itemize}
        \experience{Development Intern}{Summer 2016}{Supervisor: Prof. Manindra Agarwal, IIT Kanpur}
      \begin{itemize}
        \item Worked on a scalable web application with a diverse technology stack
        \item Used Scala with Akka and Couchbase among other technologies for developing the backend
        \item Internship was under the NYC Office of IIT Kanpur
      \end{itemize}
      \experience{Smart Mirror}{Summer 2016}{Programming Club IIT Kanpur}
      \begin{itemize}
        \item A mirror to get you ready for the day.
        \item Chosen as the \textbf{Best Applicative Project - SnT Summer Camp 2016}
        \item Link : \href{https://pclub.in/project/2016/07/06/smartmirror.html}{Smart Mirror}
      \end{itemize}
      \experience{Reversi game in Python}{$2^{nd}$ Semester}{ACA Semester Project}
      \begin{itemize}
        \item Developed a Python Application for 2 player as well as single player Reversi gameplay in a team of 2
        \item Uses the basic minimax algorithm with an efficient heuristic check for better performance against humans
        \item Mid Semester project under the Association of Computing Activities (ACA), IIT Kanpur
        \item Link : \href{http://github.com/yashsriv/Reversi-Python}{Reversi}
      \end{itemize}
      \experience{\href{http://students.iitk.ac.in/robocon/}{Robocon 2016}}{Oct'2015 - Mar'2016}{Supervisor : Prof. Bhaskar Dasgupta (IIT Kanpur)}
      \begin{itemize}
        \item Developed two robots out of which one was autonomous on a game field consisting of ramps \& turns. The autonomous robot, which did not contain a driving actuator had to traverse the game field using the energy provided to it by other robot in form of a non contact force.
        \item I was involved in \textbf{Image Processing} used in the autonomous robot for \textbf{color detection} and \textbf{line following} to traverse the arena
        \item Came \textbf{3rd} out of 105 teams participating in Nationals at Pune, India
      \end{itemize}
      \experience{Code.Fun.Do}{Sep'2015}{Microsoft India 24 Hour Hackathon}
      \begin{itemize}
        \item Developed an App to help connect teachers and learners
        \item Used cross-platform \textbf{Universal App Platform} for Windows 10
        \item Was selected as one of the best five ideas
      \end{itemize}
    \end{itemize}
  \section{Technical Skills}
  \begin{tabular}{L{0.05cm} L{6cm} l}
      & Computer Languages   & C/C++, C\#(Beginner), Java, Python, Javascript, Scala\\
      & Tools                & Git, Vim, \LaTeX, SQL, Couchbase, MongoDB, nodejs \\
      & Operating Systems    & Windows, Linux(Debian, Ubuntu, Arch)\\
      & App Development      & Windows, Android \\
      & Miscellaneous        & OpenCV, Visual Studio, AI and Game Theory
    \end{tabular}
  \section{Other Projects}
  \begin{itemize}
    \experience{Google DevFest}{Oct'2016}{Google 24 Hour Hackathon}
    \begin{itemize}
      \item Developed a simple Android App which acts as a WebSocket Client for a WebSocket Server for the Real Life Game Mafia
    \end{itemize}
    \experience{Antaragni 16 WebApp}{2016}{Nodejs backend for a fest registration portal}
    \begin{itemize}
      \item Backend written in nodejs and mongodb
    \end{itemize}
  \end{itemize}
  \section{Other Interests}
    \begin {itemize}
      \item Web Development
      \item Image Processing
      \item Artificial Intelligence
      \item Robotics
    \end{itemize}
\end{document}
