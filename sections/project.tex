\cvsection{Projects}

\begin{cventries}

  \cventry
  {Inter IIT Tech Meet}
  {Dashboard Web Extension/App}
  {IIT Kanpur}
  {February, 2017}
  {
    \begin{cvitems}
    \item Involved creating a web extension which acted as a user’s home page
      and helped display all the information relevant to a student studying in
      IIT Kanpur in a main Dashboard. (News, Events, Student Search, Share Auto)
    \item Server Side was implemented as multiple \textbf{microservices} in
      various languages (Golang, Python, NodeJS) involving IPC via \textbf{JSON
        RPC}.
    \item Fully \textbf{Dockerized backend} running on a docker-compose setup.
      Written with \textbf{scalability} in mind.
      \href{https://github.com/yashsriv/beethoven}{(github.com/yashsriv/beethoven)}
    \item Judged \textbf{1\textsuperscript{st}} among all the IITs participating
      in the competition.
    \end{cvitems}
  }

  \cventry
  {Programming Club}
  {\href{http://pclub.in/project/2016/07/06/smartmirror.html}{Smart Mirror}}
  {IIT Kanpur}
  {Summer'2016}
  {
    \begin{cvitems}
    \item Built an \textbf{IoT Mirror} with an RPi and a display fitted with a 75\%
      reflecting mirror.
    \item The mirror had features such as weather forecast, calendar
      and pushbullet notifications of a user (determined via face
      identification).
    \item Received \textbf{Best Applicable Project} amongst all summer projects under
      the Science and Technology Council, IIT Kanpur.
    \end{cvitems}
  }

  \cventry
  {Member, Team Robocon IIT Kanpur, Prof. Bhaskardas Gupta}
  {ABU Robocon 2016}
  {IIT Kanpur}
  {Oct'2015 - Mar'2016}
  {
    \begin{cvitems}
      \item An autonomous robot, which did not contain a driving actuator had to
        traverse a game field using the energy provided to it by another robot in
        form of a non contact force.
      \item I was involved in \textbf{Image Processing} used in the autonomous
        robot for \textbf{color detection} and \textbf{line following} to
        traverse the arena.
      \item Came \textbf{3\textsuperscript{rd}} out of 105 teams participating in Nationals at Pune, India.
    \end{cvitems}
  }

  \cventry
  {Senior Web Executive}
  {Antaragni'16}
  {IIT Kanpur}
  {July'2016 - Oct'2016}
  {
    \begin{cvitems}
    \item Used the full \textbf{MEAN Stack} for a fest webapp and its admin control panel.
    \item Dynamic website with content easily changeable via the control panel.
    \item Supported Android App as well with an API.
    \end{cvitems}
  }

  \cventry
  {Association of Computing Activities}
  {\href{http://github.com/yashsriv/Reversi-Python}{Reversi Game in Python}}
  {IIT Kanpur}
  {2\textsuperscript{nd} Semester}
  {
    \begin{cvitems}
    \item Developed a Python Application using \textbf{Pygame} for 2 player as well as
      single player Reversi gameplay.
    \item Uses the \textbf{negamax algorithm} with an efficient heuristic check
      for better performance against humans.
    \item Mid Semester project under the Association of Computing Activities (ACA), IIT Kanpur.
    \item Link: \href{https://github.com/yashsriv/Reversi-Python}{github.com/yashsriv/Reversi-Python}
    \end{cvitems}
  }

  \cventry
  {24 Hour Hackathon}
  {Code.Fun.Do}
  {Microsoft India}
  {Sept'2015}
  {
    \begin{cvitems}
    \item Developed an App to help connect teachers and learners.
    \item Used cross-platform \textbf{Universal App Platform} for Windows 10
      and a server written in C\#.
    \item Was selected as one of the \textbf{best five ideas}.
    \end{cvitems}
  }

\end{cventries}

%%% Local Variables:
%%% mode: latex
%%% End: