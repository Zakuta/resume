\section*{\sc Projects}
\vspace{-2mm}
\hrulefill
\vspace{.2cm}

\cventry
{Inter IIT Tech Meet}
{Dashboard Web Extension/App}
{IIT Kanpur}
{February, 2017}
{
  \begin{itemize}
  \item Involved creating a web extension which acted as a user’s home page
  \item Server Side was implemented as multiple \textbf{microservices} in
    various languages (Golang, Python, NodeJS) with client side in Angular.
  \item Fully \textbf{Dockerized scalable backend} running on a docker-compose
    setup. \\
    \href{https://github.com/yashsriv/beethoven}{(github.com/yashsriv/beethoven)}
  \item Judged \textbf{1\textsuperscript{st}} among all the IITs participating
    in the competition.
  \end{itemize}
}

\cventry
{Member, Team Robocon IIT Kanpur, Prof. Bhaskardas Gupta}
{ABU Robocon 2016}
{IIT Kanpur}
{Oct'2015 - Mar'2016}
{
  \begin{itemize}
  \item An autonomous robot, which did not contain a driving actuator had to
    traverse a game field using the energy provided to it by another robot in
    form of a non contact force.
  \item I was involved in \textbf{Image Processing} used in the autonomous
    robot for \textbf{color detection} and \textbf{line following} to
    traverse the arena.
  \item Came \textbf{3\textsuperscript{rd}} out of 105 teams participating in Nationals at Pune, India.
  \end{itemize}
}

\smallcventry
{Programming Club}
{\href{http://pclub.in/project/2016/07/06/smartmirror.html}{Smart Mirror}}
{Best Applicable Project}
{IIT Kanpur}
{Summer'2016}

\smallcventry
{24 Hour Hackathon}
{Code.Fun.Do}
{Best 5 Ideas}
{Microsoft India}
{Sept'2015}

\smallcventry
{Association of Computing Activities}
{\href{http://github.com/yashsriv/Reversi-Python}{Reversi Game in Python}}
{}
{IIT Kanpur}
{2\textsuperscript{nd} Semester}
